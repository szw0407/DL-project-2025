\documentclass{article}

% 中文支持设置
\usepackage{xeCJK}
\usepackage{indentfirst} % Add indentfirst package for first paragraph indentation
\setlength{\parindent}{2em} % Set paragraph indentation to 2em
\newCJKfontfamily\heiti[BoldFont=SimHei]{SimHei} % 设置黑体字体命令
\newCJKfontfamily\kaishu{KaiTi} % 设置楷体字体命令
\newCJKfontfamily\fangsong{FangSong} % 设置仿宋字体命令
\newCJKfontfamily\lishu{LiShu} % 设置隶书字体命令

% 页面设置
\usepackage[a4paper,top=2cm,bottom=2cm,left=3cm,right=3cm,marginparwidth=1.75cm]{geometry}

% 有用的包
\usepackage{amsmath}
\usepackage{amssymb}
\usepackage{graphicx}
\usepackage{longtable}
\usepackage{array}
\usepackage{booktabs}
\usepackage{xcolor}
\usepackage{colortbl}
\usepackage[colorlinks=true, allcolors=blue, unicode]{hyperref} % 添加 unicode 选项
\usepackage{listings}
\usepackage{float} % 添加 float 宏包
\usepackage{keywords} % 引入关键词环境定义
\usepackage{codestyle} % 引入代码样式定义

% 中文标签设置
\renewcommand{\abstractname}{摘要}
\renewcommand{\contentsname}{目录}
\renewcommand{\listfigurename}{插图目录}
\renewcommand{\listtablename}{表格目录}
\renewcommand{\refname}{参考文献}
\renewcommand{\indexname}{索引}
\renewcommand{\figurename}{图}
\renewcommand{\tablename}{表}
\renewcommand{\appendixname}{附录}

\title{深度学习报告}

\usepackage{hyperref}
\begin{document}
\maketitle

\begin{abstract}
      % TODO: 稍后补充
\end{abstract}

\newpage
% 目录

\tableofcontents
\newpage
\section{引言}
在现代商业环境中,选址决策是品牌扩张和市场布局过程中至关重要的一环。一个优质的门店位置不仅能够提升品牌曝光度和客流量,还直接影响企业的经济效益和市场竞争力。传统的选址方法主要依赖于专家经验和简单的人口统计分析,存在主观性强、难以量化、适应性差等局限。随着大数据、机器学习和深度学习技术的快速发展,基于数据驱动的选址预测方法为选址预测提供更多的思路启发,其能够通过对历史数据的深入分析,挖掘出潜在的空间模式和内在偏好,从而实现更科学、精准、有效的选址决策。

本课题聚焦于“商业智能选址预测”,旨在通过分析品牌历史门店的地理分布数据,建立能够预测品牌下一家门店最优选址的模型。具体而言,课题以网格(Grid)为基本地理单元,利用品牌在城市中的历史开店网格序列及每个网格的地理属性特征,挖掘品牌扩张的空间模式和内在偏好。通过对历史数据的建模与学习,预测品牌未来最有可能选址的网格位置,为企业提供科学、量化的决策依据。

本项目的数据集包含多个品牌的历史门店分布(训练集和测试集),以及覆盖研究区域的网格地理坐标信息。提供数据已划分为训练集和测试集,网格划分保证了空间分析的精度和一致性。研究区域的经纬度范围明确,便于空间特征的提取和建模。

在方法设计上,项目不仅实现了数据预处理、特征增强、模型训练与评估等完整流程,还探索了多种深度学习与空间分析方法,包括序列建模(如RNN、LSTM、Transformer)、空间关系建模(如图神经网络GNN)以及多模态信息融合等。针对品牌门店分布的空间依赖性和数据稀疏性等挑战,项目尝试引入迁移学习等先进技术,以提升模型的泛化能力和预测准确率。

\section{相关工作}

项目提供多种基线(Baseline)模型或方法,包括简单模型和深度模型。传统的随机猜测(Random guess)方法将所有未出现过的网格作为候选集,随机选取若干作为预测结果,作为最简单的基线。随着空间数据挖掘的发展,基于Word2vec的嵌入方法被引入到选址任务中。该方法通过skip-gram模型训练每个网格的embedding,测试时将品牌历史网格序列的embedding取平均,计算与候选网格embedding的相似度(如点积、余弦相似度等),并据此排序预测结果。

深度学习方法方面,LSTM+MLP结构利用嵌入层将输入网格序列编码为向量,随后通过LSTM捕捉序列中的时序依赖,最后通过多层感知机(MLP)进行分类预测。Transformer+MLP则用Transformer替代LSTM,利用自注意力机制更好地建模序列中各网格之间的复杂关系。进一步地,基于注意力融合和对比学习的方法,通过对输入序列进行self-attention,获得全局语义表达,并采用InfoNCE损失进行正负样本的对比训练,有效提升了模型的判别能力和泛化性能。

上述方法在本项目数据集上的实验结果如表所示。可以看出,深度学习模型(如Transformer+MLP、对比学习方法)在准确率和平均排序等指标上均优于传统方法,尤其是对比学习方法在acc@1和acc@10等指标上表现突出,显示了其在空间选址预测任务中的潜力。此外,模型结构的层级设计(如嵌入层、序列建模层、融合层和输出层)对最终性能有显著影响,合理的结构选择和层级组合能够更好地捕捉空间和序列特征,从而提升预测效果。

% TODO:这部分内容后面重写

\section{方法探索}

\subsection{数据预处理}

在商业选址预测任务中,原始数据通常包含品牌历史门店的网格ID序列、网格地理坐标以及兴趣点(POI)特征等多种异构信息。为了使这些数据能够有效地用于深度学习模型训练,我们设计了一套完整的数据预处理流程。

\subsubsection{网格信息加载与归一化}

首先,我们从网格坐标文件中提取每个网格的地理信息,包括经纬度坐标和10种类型的POI特征(医疗、住宿、摩托、体育、餐饮、公司、购物、生活、科教、汽车)。为了消除不同特征间的尺度差异,我们对坐标和POI特征分别进行归一化处理:

对于坐标特征,我们计算网格的中心点坐标,然后将其归一化到[0,1]区间:
\begin{equation}
x_{norm} = \frac{x - x_{min}}{x_{max} - x_{min} + \epsilon}
\end{equation}

对于POI特征,我们同样采用最小-最大归一化方法,确保不同类型POI特征的数值在同一尺度上。

\subsubsection{基于密度的序列排序}

考虑到商业扩张通常遵循"从高密度区域向外扩展"的规律,我们设计了基于密度的序列排序算法。该算法通过计算每个网格到其K个最近邻网格的平均距离来衡量密度,距离越小表示密度越大:

\begin{equation}
density\_score_i = \frac{1}{K}\sum_{j=1}^{K} d(grid_i, neighbor_j)
\end{equation}

其中$d(grid_i, neighbor_j)$表示网格$i$到其第$j$个最近邻的欧氏距离。通过这种排序方式,我们能够构建符合商业扩张规律的序列,提高模型的学习效果。

\subsubsection{滑动窗口样本生成}

为了最大化利用有限的训练数据,我们采用滑动窗口方法从每个品牌的店铺序列中生成多个训练样本。例如,对于序列[A,B,C,D],我们生成以下样本:
\begin{itemize}
\item 已有[A],预测B
\item 已有[A,B],预测C  
\item 已有[A,B,C],预测D
\end{itemize}

这种处理方法不仅增加了训练样本数量,还使模型能够学习不同长度序列的预测模式,更好地模拟品牌实际扩张过程中的决策链。

% TODO: 用户可在此补充更多数据预处理的尝试和优化

\subsection{对于标签的处理尝试}

% TODO: 用户可在此补充对标签处理的具体尝试,如类别不平衡处理、标签平滑等

\subsection{数据增强和特征增强}

% TODO: 用户可在此补充数据增强的具体方法,如噪声注入、序列扰动、特征工程等

\subsection{多模态特征融合设计}

针对商业选址预测任务的特点,我们设计了一个多模态融合的神经网络架构,该架构能够同时处理和融合三种不同类型的信息:

\subsubsection{序列特征编码}
历史选址序列包含了品牌扩张的时序信息和空间模式。我们使用嵌入层将网格ID映射到低维向量空间,然后通过LSTM网络捕获序列中的时序依赖关系:

\begin{equation}
h_t = LSTM(embed(grid\_id_t), h_{t-1})
\end{equation}

LSTM的长短期记忆特性使其能够有效学习品牌在不同时期的选址偏好变化。

\subsubsection{空间特征编码}
地理坐标信息反映了选址的空间连续性。我们通过多层感知机(MLP)对归一化后的坐标特征进行编码,提取空间分布模式。

\subsubsection{环境特征编码}
POI特征反映了网格周边的商业环境和设施分布。同样使用MLP对POI特征向量进行编码,捕获环境因素对选址决策的影响。

\subsubsection{特征融合策略}
我们采用特征拼接的方式将三种编码后的特征向量组合,然后通过多层融合网络进一步提取联合特征:

\begin{equation}
f_{fused} = MLP_{fusion}(concat(f_{seq}, f_{coord}, f_{poi}))
\end{equation}

最终通过线性分类器预测下一个网格的概率分布。

\subsection{模型最终设计}

考虑到数据集中特征的特点,模型采用时序的方案进行建模。整体架构采用多模态融合设计,具体包括:

\begin{itemize}
\item \textbf{序列编码器}:基于嵌入层和LSTM的序列特征提取
\item \textbf{空间编码器}:基于MLP的坐标特征编码
\item \textbf{环境编码器}:基于MLP的POI特征编码  
\item \textbf{融合网络}:多层感知机进行特征融合
\item \textbf{分类器}:线性层输出概率分布
\end{itemize}

该设计充分考虑了商业选址的多因素特性,通过端到端的学习方式自动发现不同模态特征间的内在关联,为精准的选址预测提供了有力支撑。

% TODO: 用户可在此补充模型设计的其他尝试,如不同架构的对比、超参数调优等

\appendix
\section{附录}

\subsection{项目信息}
\textbf{GitHub仓库地址:} \url{https://github.com/szw0407/DL-project-2025}

\subsection{核心模型代码}
\subsubsection{神经网络模型实现 (model.py)}
\lstinputlisting[language=Python, caption={多模态神经网络模型实现}, label={lst:model}]{../src/model.py}

\subsubsection{主程序入口 (main.py)}
\lstinputlisting[language=Python, caption={主程序实现}, label={lst:main}]{../src/main.py}

\subsubsection{训练逻辑实现 (train.py)}
\lstinputlisting[language=Python, caption={模型训练实现}, label={lst:train}]{../src/train.py}

\subsubsection{数据预处理模块 (data\_preprocessing.py)}
\lstinputlisting[language=Python, caption={数据预处理实现}, label={lst:preprocessing}]{../src/data_preprocessing.py}

\subsubsection{数据特征增强 (测试数据文件.py)}
\lstinputlisting[language=Python, caption={数据特征增强实现}, label={lst:test_data}]{../src/测试数据文件.py}

\subsubsection{模型评估模块 (evaluate.py)}
\lstinputlisting[language=Python, caption={模型评估实现}, label={lst:evaluate}]{../src/evaluate.py}

\subsection{数据样本展示}
\subsubsection{训练数据格式}
以下直接展示训练数据文件的前10行内容:
\inputdatafile{../data/train_data.csv}{训练数据样本 (train\_data.csv)}{lst:train_data}

\subsubsection{网格坐标映射}
以下直接展示网格坐标映射文件的前10行内容:
\inputdatafile{../data/grid_coordinates-2.csv}{网格坐标映射数据 (grid\_coordinates-2.csv)}{lst:grid_data}

\subsubsection{测试数据格式}
以下直接展示测试数据文件的前10行内容:
\inputdatafile{../data/test_data.csv}{测试数据样本 (test\_data.csv)}{lst:test_data}

\subsection{完整源代码文件结构}

项目包含以下主要文件:

\begin{itemize}
    \item \texttt{src/model.py} - 神经网络模型定义
    \item \texttt{src/main.py} - 主程序入口
    \item \texttt{src/train.py} - 训练逻辑实现
    \item \texttt{src/evaluate.py} - 模型评估
    \item \texttt{src/data\_preprocessing.py} - 数据预处理
    \item \texttt{data/train\_data.csv} - 训练数据
    \item \texttt{data/test\_data.csv} - 测试数据
    \item \texttt{data/grid\_coordinates-2.csv} - 网格坐标映射
\end{itemize}

详细的代码实现和更多技术细节请参考GitHub仓库:\url{https://github.com/szw0407/DL-project-2025}

\end{document}