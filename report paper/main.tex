\documentclass{article}

% 中文支持设置
\usepackage{xeCJK}
\usepackage{indentfirst} % Add indentfirst package for first paragraph indentation
\setlength{\parindent}{2em} % Set paragraph indentation to 2em
\setCJKmainfont[BoldFont=SimHei,ItalicFont=KaiTi,AutoFakeBold=true,BoldItalicFont=SimHei]{SimSun} % 设置中文字体为宋体,粗体使用黑体
\setCJKsansfont[BoldFont=SimHei]{SimHei} % 设置中文无衬线字体为黑体
\setCJKmonofont[BoldFont=SimHei]{SimSun} % 设置中文等宽字体
\newCJKfontfamily\heiti[BoldFont=SimHei]{SimHei} % 设置黑体字体命令
\newCJKfontfamily\kaishu{KaiTi} % 设置楷体字体命令
\newCJKfontfamily\fangsong{FangSong} % 设置仿宋字体命令
\newCJKfontfamily\lishu{LiShu} % 设置隶书字体命令

% 页面设置
\usepackage[a4paper,top=2cm,bottom=2cm,left=3cm,right=3cm,marginparwidth=1.75cm]{geometry}

% 有用的包
\usepackage{amsmath}
\usepackage{amssymb}
\usepackage{graphicx}
\usepackage{longtable}
\usepackage{array}
\usepackage{booktabs}
\usepackage{xcolor}
\usepackage{colortbl}
\usepackage[colorlinks=true, allcolors=blue, unicode]{hyperref} % 添加 unicode 选项
\usepackage{listings}
\usepackage{float} % 添加 float 宏包
\usepackage{keywords} % 引入关键词环境定义
\usepackage{codestyle} % 引入代码样式定义

% 中文标签设置
\renewcommand{\abstractname}{摘要}
\renewcommand{\contentsname}{目录}
\renewcommand{\listfigurename}{插图目录}
\renewcommand{\listtablename}{表格目录}
\renewcommand{\refname}{参考文献}
\renewcommand{\indexname}{索引}
\renewcommand{\figurename}{图}
\renewcommand{\tablename}{表}
\renewcommand{\appendixname}{附录}

\title{深度学习报告}

\usepackage{hyperref}
\begin{document}
\maketitle

\begin{abstract}
      TODO: 稍后补充
\end{abstract}

\newpage
% 目录

\tableofcontents
\newpage
\section{引言}
在现代商业环境中,选址决策是品牌扩张和市场布局过程中至关重要的一环。一个优质的门店位置不仅能够提升品牌曝光度和客流量,还直接影响企业的经济效益和市场竞争力。传统的选址方法主要依赖于专家经验和简单的人口统计分析,存在主观性强、难以量化、适应性差等局限。随着大数据、机器学习和深度学习技术的快速发展,基于数据驱动的选址预测方法逐渐成为学术界和业界关注的热点。

本课题聚焦于“商业智能选址预测”,旨在通过分析品牌历史门店的地理分布数据,建立能够预测品牌下一家门店最优选址的模型。具体而言,课题以网格(Grid)为基本地理单元,利用品牌在城市中的历史开店网格序列及每个网格的地理属性特征,挖掘品牌扩张的空间模式和内在偏好。通过对历史数据的建模与学习,预测品牌未来最有可能选址的网格位置,为企业提供科学、量化的决策依据。

本项目的数据集包含多个品牌的历史门店分布(训练集和测试集),以及覆盖研究区域的网格地理坐标信息。数据已按7:3比例划分为训练集和测试集,网格划分保证了空间分析的精度和一致性。研究区域的经纬度范围明确,便于空间特征的提取和建模。

在方法设计上,项目不仅实现了数据预处理、特征增强、模型训练与评估等完整流程,还探索了多种深度学习与空间分析方法,包括序列建模(如RNN、LSTM、Transformer)、空间关系建模(如图神经网络GNN)以及多模态信息融合等。针对品牌门店分布的空间依赖性和数据稀疏性等挑战,项目尝试引入迁移学习等先进技术,以提升模型的泛化能力和预测准确率。

本课题的研究不仅具有重要的理论意义,也为实际商业选址提供了可行的智能化解决方案。通过本项目的实践,同学们能够深入理解数据驱动的空间决策建模流程,掌握深度学习与空间分析的前沿方法,并积累解决真实复杂问题的宝贵经验。

\appendix
\section{附录}

\subsection{项目信息}
\textbf{GitHub仓库地址:} \url{https://github.com/szw0407/DL-project-2025}

\subsection{核心模型代码}
\subsubsection{神经网络模型实现 (model.py)}
\lstinputlisting[language=Python, caption={多模态神经网络模型实现}, label={lst:model}]{../src/model.py}

\subsubsection{主程序入口 (main.py)}
\lstinputlisting[language=Python, caption={主程序实现}, label={lst:main}]{../src/main.py}

\subsubsection{训练逻辑实现 (train.py)}
\lstinputlisting[language=Python, caption={模型训练实现}, label={lst:train}]{../src/train.py}

\subsubsection{数据预处理模块 (data\_preprocessing.py)}
\lstinputlisting[language=Python, caption={数据预处理实现}, label={lst:preprocessing}]{../src/data_preprocessing.py}

\subsubsection{数据特征增强 (测试数据文件.py)}
\lstinputlisting[language=Python, caption={数据特征增强实现}, label={lst:test_data}]{../src/测试数据文件.py}

\subsubsection{模型评估模块 (evaluate.py)}
\lstinputlisting[language=Python, caption={模型评估实现}, label={lst:evaluate}]{../src/evaluate.py}

\subsection{数据样本展示}
\subsubsection{训练数据格式}
以下直接展示训练数据文件的前10行内容:
\inputdatafile{../data/train_data.csv}{训练数据样本 (train\_data.csv)}{lst:train_data}

\subsubsection{网格坐标映射}
以下直接展示网格坐标映射文件的前10行内容:
\inputdatafile{../data/grid_coordinates-2.csv}{网格坐标映射数据 (grid\_coordinates-2.csv)}{lst:grid_data}

\subsubsection{测试数据格式}
以下直接展示测试数据文件的前10行内容:
\inputdatafile{../data/test_data.csv}{测试数据样本 (test\_data.csv)}{lst:test_data}

\subsection{完整源代码文件结构}

项目包含以下主要文件:

\begin{itemize}
    \item \texttt{src/model.py} - 神经网络模型定义
    \item \texttt{src/main.py} - 主程序入口
    \item \texttt{src/train.py} - 训练逻辑实现
    \item \texttt{src/evaluate.py} - 模型评估
    \item \texttt{src/data\_preprocessing.py} - 数据预处理
    \item \texttt{data/train\_data.csv} - 训练数据
    \item \texttt{data/test\_data.csv} - 测试数据
    \item \texttt{data/grid\_coordinates-2.csv} - 网格坐标映射
\end{itemize}

详细的代码实现和更多技术细节请参考GitHub仓库:\url{https://github.com/szw0407/DL-project-2025}

\end{document}